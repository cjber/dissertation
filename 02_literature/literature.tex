\begin{multicols}{2}
\section{Literature Review}
\label{sec:literature}

\subsection{Road Extraction Techniques}

\lettrine{T}{ypical} road extraction techniques have focused purely on rural road networks, where techniques involving LiDAR often rely on the kerb and change in height to determine road edges.

\subsection{National Speed Limits}

Accidents on rural roads often occur within the 60 mph speed limit meaning a distinction between what is an appropriate speed should be made that does not relate to a given speed limit. \cite{baruya1998} suggest a distinction between both \textit{excess} and \textit{inappropriate} speed. \textit{Excess} speed being driving above the speed limit, and therefore directly breaking the law, \textit{inappropriate} speed; driving too fast for the conditions of the road, not necessarily above the speed limit.

Lowering the speed limit on roads has been shown to result in an overall reduction in the average speed of vehicles \citep{finch1994}, and subsequently the number of road traffic accidents \citep{taylor2002}. Additionally, \cite{taylor2000} found that accident frequency is linked with traffic and pedestrian flow, vehicle speed, and particular features of the road geometry.

As noted with the MASTER European speed study, access to quality sources of data which includes the road geometry for use in individual road assessments has been limited. \cite{yadev2018} note that the periodic assessment of roads is greatly important due to the increasing traffic load, and new automated techniques will enable this in areas where in the past it had not been feasible.

\subsection{Accessibility}

Transport accessibility for rural communities is far more limited than for urban communities, where often rural areas have limited or no public transport, meaning there is a heavy reliance on personal transport \citep{gray2001}. Accessibility in this context can be defined as the transport facility or opportunities with which basic services can be reached from a given location by using a certain transport system \citep{gutierrez2009}.

There has been little focus on the improvement of transport technologies in rural areas, with the potential for new technologies implemented into urban areas improving rural accessiblity \citep{velaga2012}. A key area to address is the level of accessibility to hospital services for rural communities, where recent centralisation of these services has negatively impacted the level of access for rural communities \citep{mungall2005}. This also impacts the level of access for hospital services to reach rural areas, where distance and time taken to a hospital directly correlates with a patients mortality \citep{nicholl2007}. Emergency vehicles are often larger than personal vehicles and as such it can be assumed that accessibility for these types of services is often more limited depending on the quality of rural roads.

\section{Feasibility for Road Extraction in the UK}

Traditional road extraction techniques from aerial imagery relied on human extraction that is both time consuming and prone to inaccuracy \citep{wang2015}. However, recent improvements in remote sensing technology now mean that automated extraction techniques have been explored which rely on the much higher resolution imagery available \citep{xu2018}. This paper will use Light Detection and Ranging (LiDAR) data, made available by the UK government for much of England. LiDAR is collected by measuring the distance from an aircraft to the ground, emitting laser pulses at regular intervals \citep{environmentagency2019}, allowing for very high resolution terrain models to be produced, with a resolution up to 25cm. Additionally the 'point clouds' produced by the laser pulses contain information that may be utilised to determine the qualities that are indicative of road locations \citep[e.g.][]{clode2004}.

Additionally, luminescence information extracted from 25cm resolution aerial true colour imagery is available for research purposes for the entirety of the UK.

// timeline of aerial image quality /lidar

See \citep{kumar2013}:

vosselman2009: segmenting lidar data identifivation of planar or smooth surfaces and classification of point cloud data based on attributes.
clode2004: segmentation of airborne lidar to road and non-road objects using hierarchical classification techniiques based on elevation and intensity information, accuracy impaired due to car parks and private roads.
goulette2006: methodd for segmenting road, trees, and facades from terrestrial mobile lidar data. road sgemented as horizontal plane high density of points in the histogram + usedt o compute road width and curvature. Trees and facades identified as vertical planes and disconnected elements in the histogram.
yuan2008: algoritm using fuzzy clusting method to cluster LiDAR points, straight lines then fitted to the linearly clustered data using slope information for extracting the road.
elberink2009: automated method for 3d modelling highway infrastructure using airborne LiDAR + 2d topographic. Road polygons were extracted from the topographic map data using a map based seed-growing algorithm combined with a Hough transformation. points added to the correspondng road polygons using a LiDAR based seed-growing algorithm then mapped polygons

Most of the proposed methods (Ferchichi and Wang, 2005; Wan et al., 2007; Mokhtarzade and Valadan Zoej, 2007; Mena and Malpica, 2005; Mohammadzadeh et al., 2006; Wang et al., 2005) using satellite images and aerial photographs only provide road pixels and its two- dimensional (2D) location information.

- Extraction includes detection of planar or smooth surfaces, and the classification of points or point clusters based on local point patterns, echo intensity and echo count information \citep{vosselman2004;darmawati2008}

2. Lidar points distirbute irregularly with non uniformity density. the point density in the overlapping area between fight strips is greater than non-overlapping regions. There is no point in areas occulded by tall objects. can be soled interpolating into intensity image (e.g. Zhao,2012, zhu 2009) binary image, Clode 2007. others see Samadzadegan 2009, Jiangui 2011.

\citep{yadav2018}: Major limitation of existing methods to extract rural road surface as mostly without curb. Methologies such as Yang et al., 2013 which deals with curbs and boundaries characterized as ashpalt/soil and ashpalt vegetation and ashpalt/grassy bank. Exact edge location difficult if point density less and sufficiant elevatio jump not present at boundary of carriageway (Mc Elhinney et al., 2010). Wu et al., 2013 dependent on road range which cannot be calculated correctly if height diffrernce between road surface and shoulder not obvious.



- from \citep{vosselman2009a}: clode 2004a, pulse reflectance strength + dtm to detect roads . Later improved including buildings and vegetation.

- Akel et al. 2005, road areas classifying smooth segments based on size and shape characteristics
- Rieger 1999 used surface slope in a 20cm eleveation grid to detect roads in mountainous areas
- Hatger and Brenner 2003,2005, laser scanning data at 0.5m to extract properties of roads for which centrelines arer known from a road database. Including height longitudinal and transversal slope, curvature, and width. Height profiles across road split into straight sgements. A RANSAC procedure on the end points of the sements with low slope values is used to fit straight lines to the road sides. \citep{hatger2003;hatger2005}

Look at \citep{taylor2002} for a full assessment of rural roads. Includes bends, hills etc etc.

Clode 2004:

Note that as they use point density of 1/2m, and intensity has a footprint of 20 - 30cm the intensity if not typically used. (Rottensteiner, 2003). **with my method we use a resolution of 25cm, much higher point density**


// list varying resolution increases per papers?

From: \citep{hatger2005}
- Segmentation through discontinuity, prominent oints or edges, or areas fulfilling certain criteria in continuity.
- Point operators which try to find isolated points, corners, or points being part of a one dimensional curve (Haralick and Shapira, 1992, Canny, 1986 (old))
- Second step building contour chains from individaul points.
- alternatively, discontinuity perpendicular to a linear structure can be defined in terms of the continuous arass to the left and right of the structure (brugelmann,2000, wild and krzystek 1996. + region growing)

- inclusion of as much information e.g. minimum width requirements etc (actually not defined in the UK however)
- Continuous surfaces approximated by planes, usually horizontal

Techniques from Hatger 2005:o
General Planar Regin growsing Segmentation
Iterating following:
1. Find the best seed regin which fulfils the desired predicate,
2. Add elements to the seed region (i.e. grow it) as long as they are connected to it, and they too fulfil the predicate
3. If the region cannot be grown anymore, accept it and go to 1. using the remaining elements.
They use estimation of local planes, and look for the smallest residuals. The predicate is a certain maximum distance $e$ of the points to the plane given by its Hesse normal form

The scan line grouping approach
Less computationally expensive than region growing.
- Presented by \citep{jiang1992}.
- Raster data: plane $z = ax + by + d$ all points along the line $y = y0$ fulful the equatin $z = as + by0 + d$ the line equation. add entire scan lines
Steps:
1. Partitioning of each scan line y = y0 into linear segments fulfilling corresponding line equaitons
2. Search for a seed region by investigating overlapping linear seegments of three successive lines
3. Growing the best seed region by adding neighbouring line segments, as long as they still are part of the same plane
4. Postprocessing: 

(RANSAC), intoduced by (Fischler and Bolles, 1981). fixes errors

from Clode 2004:

Road pixel must lie on or near DTM, and have a certain intensity and normalised local point density. **My paper will extend this, inclusion of road shapefiles to remove known non roads, noted as a limitation in clode therefore, less focus on automated extraction, (as it isn't needed generally), more focus on automated identification of road features, i.e. width etc.**

Intensity identifies the type of road material (bitumen in their paper)

Create DTM from last pulse DSM **may not be need for my work if I am already filtering onto known approximate locations of roads**

Intensity identifies the type of road material (bitumen in their paper)

Charainya gives information which is utilised in this paper for the modelling. i.e. Lum and Intensity are most useful for road detection

\section{Rural Roads and Speed}

112. Rural roads account for 66\% of all road deaths, and 82\% of car
occupant deaths in particular, but only around 42\% of the distance
travelled. Of all road deaths in Britain in 2011, 51\% occurred on National
Speed Limit rural single carriageway roads (DfT, 2011). The reduction in
road casualties and especially deaths on rural roads is one of the key road
safety challenges. Research has assessed the risk of death in collisions at
various impact speeds for typical collision types on rural roads. This
research suggests that the risk of a driver dying in a head on collision
involving two cars travelling at 60 mph is around 90\%, but that this dropsrapidly with speed, so that it is around 50\% at 48 mph (Richards and Cuerden, 2009).

113. Inappropriate speed, at levels below the legal limit but above those
appropriate for the road at the time (for example, because of the weather
conditions or because vulnerable road users are present), is a particular
problem for rural roads. Exceeding the speed limit or travelling too fast for
the conditions are reported as contributory factors in 16\% of collisions on
rural roads. Specifically, inappropriate speed is recorded as a contributory
factor in 20\% of crashes on minor rural roads with a 60 mph limit.
114. Speed limit changes are therefore unlikely to fully address this problem


34. On rural roads there is often a difference of opinion as to what constitutes
a reasonable balance between the risk of a collision, journey efficiency
and environmental impact. Higher speed is often perceived to bring
benefits in terms of shorter travel times for people and goods. However,
evidence suggests that when traffic is travelling at constant speeds, even
at a lower level, it may result in shorter and more reliable overall journey
times, and that journey time savings from higher speed are often
overestimated (Stradling et al., 2008). The objective should be to seek an
acceptable balance between costs and benefits, so that speed-
management policies take account of environmental, economic and social
effects as well as the reduction in casualties they are aiming to achieve.

\citep{departmentfortransport2013a} Straightforward method for determining appropriate speed limits.Forcasting speed reduction and therefore accident reduction with revised speed limits. Does not take into account road geometry etc.

Traffic Advisory Leaflet

VERALL PATTERNS OF SPEED ON DIFFERENT ROADSSpeeds on single carriageway rural roads are generally wellwithin the national 60mph speed limit. Observations from270 sites around England show a wide distribution of meanspeeds on roads with speed limits of 60mph. They alsoshow that, for the majority of roads, mean speeds arealready below the posted limit or would typically only needto be reduced by a few miles per hour, where lower limitsmight be required.

    The Transport Researech Laboratory (TRL) have produced extensive evidence that confirms the correlation between speed and the likelihood of road accidents, demonstrating that faster speeds do directly increase the chance that a driver is involved in an accident \citep{taylor2002}. Additional assessments roads within the EU have confirmed this relationship particularly focusing on rural roads with the production of the EURO model \citep{baruya1998}, however \cite{taylor2002} note that the data used in this study was very limited for the United Kingdom due to the time consuming extraction of road features, overall the study found that 20\% of fatal accidents occured on rural roads.

Key factors in determining poor road quality:

\begin{itemize}
    \item Width \citep{taylor2002} \citep{aarts2006}
    \item Surface quality
    \item Blind corners/winding roads \citep{aarts2006}
    \item Junction Sharpness \citep{aarts2006}

\end{itemize}

// read these papers and identify why these should be considered when assessing road quality
\end{multicols}
